\documentclass[a4paper, 11pt, nofonts, nocap, fancyhdr]{ctexart}

\usepackage[margin=60pt]{geometry}

\usepackage{fontspec}
\setmainfont{Arial}
\setsansfont{Arial}
\setmonofont{Consolas}

\usepackage{xeCJK}
\setCJKmainfont[BoldFont={FZDaHei-B02}, ItalicFont={FZXingKai-S04}]{FZLanTingHei-R-GBK}
\setCJKsansfont{FZLanTingHei-R-GBK}
\setCJKmonofont{FZLanTingHei-R-GBK}

\usepackage{enumitem}
\setenumerate[1]{itemsep=0pt,partopsep=0pt,parsep=\parskip,topsep=0pt}
\setitemize[1]{itemsep=0pt,partopsep=0pt,parsep=\parskip,topsep=0pt}
\setdescription{itemsep=0pt,partopsep=0pt,parsep=\parskip,topsep=0pt}

\usepackage{listings}
\lstset{
	language=SQL,
	basicstyle=\ttfamily\small,
	numbers=left,
	numberstyle=\tiny,
	breaklines=true,
	tabsize=4,
	showstringspaces=true,
	extendedchars=false
}

\usepackage{graphicx}
\usepackage{subfigure}
\usepackage{underscore}

\usepackage{ulem}
\usepackage[bookmarks=true,colorlinks,linkcolor=black]{hyperref}

\CTEXoptions[today=small]

\pagestyle{plain}

\title{MySQL查询优化}
\author
{
	梁晓涛\\
	13307130319
	\and
	李斯哲\\
	13307130370
}
\date{\today}

\begin{document}

\maketitle
\tableofcontents
\setcounter{page}{0}
\thispagestyle{empty}
\newpage

\section{实验环境}
\begin{description}
	\item[操作系统] Ubuntu 14.04
	\item[数据库] MySQL 5.5
\end{description}

\section{Task 1: 数据库设计}

\subsection{数据分析}
\begin{itemize}
	\item 删除
		\begin{itemize}
			\item 为固定值的字段: big_cate == '美食', bag_cate_id == '10', map_type == '7'
			\item 全空字段: traffic, atmosphere, payment, nearby_shops
			\item 重复字段: google_latitude == original_latitude, google_longitude == original_longitude
		\end{itemize}
	\item 函数依赖:
		\begin{itemize}
			\item city_id $\to$ city
			\item city $\to$ city_pinyin
			\item city_pinyin $\to$ city_id
			\item city $\to$ province
			\item shop_id $\to$ 所有属性
		\end{itemize}
	\item 分析各个字段的类型、长度和数值范围,以选取最合适的数据类型
	\item 根据函数依赖将原表拆分为shop和city两张表,并且两张表都符合第三范式
\end{itemize}

\subsection{建表语句}
\lstinputlisting{../SQL/createTable.sql}

\subsection{导入数据}

表city导入记录59行,表shop导入记录1000行。

\lstinputlisting{../SQL/loadData.sql}

\section{Task 2: 查询优化设计}

由于查询时间与查询时机器的各项性能有关,以下的查询时间均取多次查询中的平均时间。

\lstset{numbers=none}

\subsection{单表操作}

\subsubsection{查询表中所有字段}

	\begin{lstlisting}
	select * from city;  -- 约0.0003s
	\end{lstlisting}

由于选出全部city中的字段,所以至少读取59行。explain语句显示操作预计从59条记录中取出。

	\begin{lstlisting}
	select * from shop;  -- 约0.01s
	\end{lstlisting}

同上,至少读取1000行。explain语句显示操作预计读取1029行(这个值是估计值,每次重新连接后的值不同)。

并且这两种查询由于必须读取全部数据,因此无法进行查询优化。
	
\subsubsection{查询表中指定字段}
	
	\begin{lstlisting}
	select city_id,city from city;  -- 0.0003s
	\end{lstlisting}

需要浏览全部city元组,至少需读取59行。explain语句显示操作预计从59条记录中取出。即使对city字段加上索引也无法提高查询速度,因为无论如何都需要读取全部数据。
	
	\begin{lstlisting}
	select shop_id from shop;  -- 约0.0009s
	\end{lstlisting}

需要浏览全部元组,至少需要1000行。explain语句显示使用了shop中的主键索引shop_id,猜想采用了唯索引扫描的方法,利用索引中的数据目录项,因此不需要从关系中读取元组,查询时间相比于3.1.1中减少了很多。

	\begin{lstlisting}
	select shop_id,`name`,description from shop;  -- 约0.005s
	\end{lstlisting}

同样需要浏览1000行,由此可以看出在需要浏览相同行数的时候,返回的列数与执行时间正相关。

对于需要返回一个字段全部数据的查询,添加索引往往并不能提高查询速度,因为无论如何都需要读取全部的数据,因此在查询中应尽量避免这种查询。

\subsubsection{查询表中没有重复的字段(distinct)的使用}

	\begin{lstlisting}
	select distinct(area) from shop;  -- 约0.005s
	\end{lstlisting}

由于在area上不存在索引,因此需要浏览全部数据。explain语句显示由于distinct需要排序,因此使用了额外的表来排序,加长了查询时间。

优化:可对area字段加上索引,此时查询时间缩短到约0.003s,explain语句显示没有使用额外表来排序,而是使用了索引,预计读取行数也少于全部。

\subsubsection{条件查询各表主键的字段(单值查询或范围查询)}

	\begin{lstlisting}
	select `name` from shop where shop_id>10016726 and shop_id<10021920;  -- 约0.0007s
	\end{lstlisting}

在shop_id上存在主键,因此这样的范围查询可直接使用索引,查询时间很短。explain语句显示使用了主键索引,读取了104行数据。

在对主键字段进行查询时,因为在主键上均存在索引,这样的范围或单值查询速度会非常快。

\subsubsection{条件查询各表中普通字段(单值查询或范围查询)}
	
	\begin{lstlisting}
	select shop_id,`name` from shop where small_cate='面包甜点';  -- 约0.0015s
	\end{lstlisting}

由于在small_cate中不存在索引,在进行这样的范围查询时必须搜索全部记录,效率很低。explain语句显示查询使用了where,读取了全部数据,返回149条记录。

优化:在small_cate上添加索引,之后查询时间缩短到小于0.001s,explain语句显示查询使用了索引,且只需要读取要返回的149条记录,因此时间大大缩短。

	\begin{lstlisting}
	select shop_id,`name` from shop where avg_price<20;  -- 约0.0027s
	\end{lstlisting}

同上,在avg_price上没有索引,需要读取全部行,相比3.1.4中的范围查询慢了许多。

优化:对avg_price创建索引,查询时间缩短到0.001s,同样只需读取与返回数相同数量的元组。

由此可见,创建索引对于单值查询和范围查询有着比较好的效果,能够显著提高查询效率。

\subsubsection{一个表中多个字段条件查询(单值查询或范围查询)}
	
	\begin{lstlisting}
	select shop_id,`name` from shop where small_cate='面包甜点' and avg_price<20;  -- 约0.0013s
	\end{lstlisting}

small_cate与avg_price均不存在索引,需要读取全部行,再选出符合条件的元组,查询时间较长。

优化:对所查询的字段增加索引,当对avg_price添加索引时,查询时间缩短到0.0009s,MySQL会先对avg_price通过索引选出所用元组,再选择small_cate
若对small_cate添加索引,时间缩短到0.0007s,这是因为先选择small_cate符合条件的元组数目比avg_price小很多。若对small_cate和avg_price添加共同索引,查询时间降低到0.0006s左右,mysql直接访问索引然后读取符合条件的元组。因此对于单值或范围查询,应尽量使索引覆盖查询的字段,这样能够明显的缩短查询时间。
	
\subsubsection{一个表中group by、order by、having联合条件查询}

	\begin{lstlisting}
	select small_cate,count(*) from shop where avg_price<20 group by small_cate;   -- 约0.0015s
	\end{lstlisting}

MySQL在处理group by时会建立一个临时表。然后利用mysql内部算法,算出来结果,但由于使用了临时表和排序,查询时间通常较长。explain语句显示using temporary,using filesort。

优化:在这个查询语句之中,where部分和group by部分分别用了两个字段,因此只对其中一个字段加索引并不能有效提高查询效率,需要对avg_price,small_cate同时添加索引(avg_price,small_cate),此时查询时间缩短到0.0007s左右,explain语句显示查询使用了索引,但没有使用临时表和排序,因此查询时间大大缩短。

注意:group by对于
\begin{itemize}
	\item 对不同的索引键做group by,如 select * from a1 group by key1,key2;
	\item 在非连续的索引键部分上做 group by,如 select * from t1 where key2=constant group by key_part2;
	\item where部分与group by部分不在同一索引中
\end{itemize}
这几种情况下不会使用索引,因此在数据库索引已经创建好之后,应尽量避免这一类的查询。

	\begin{lstlisting}
	select `name`,product_rating,avg_price from shop where small_cate='面包甜点' order by avg_price,product_rating;  -- 约0.0016s
	\end{lstlisting}

在无索引情况下,MySQL执行order by语句需要对order by进行排序,比较耗费时间。explain语句显示执行操作using filesort。

优化:同group by,需要同时对where部分与order by部分添加索引,并且索引的顺序也十分重要,(avg_price,product_rating,small_cate)形式的索引几乎不能提高查询效率,而(small_cate,avg_price,product_rating)形式的索引可以将查询时间缩短到0.0009s左右。

注意:同group by,对于以上几种情况以及在order by中同时使用了desc,asc时不会用到索引,查询效率提高较少。

having和where的区别在于having可以用来筛选聚合后的数据,where在聚合之前进行筛选,这样的筛选并不会影响查询的速度,考虑以下两条语句

	\begin{lstlisting}
	select small_cate,count(*) from shop where avg_price<20 group by small_cate;
	select small_cate,count(*) from shop where avg_price<20 group by small_cate having count(*)>2;
	\end{lstlisting}

这两条语句在查询速度上几乎相同,所以优化的主要部分在前面的操作上。

\subsection{复合查询}

\subsubsection{多表联合查询及join查询}

	\begin{lstlisting}
	select shop_id,`name`,city from shop,city where shop.city_id=city.city_id and avg_price<20;  -- 约0.0024s
	\end{lstlisting}

对于上述查询,MySQL会先将shop与city做笛卡尔积,再从中选择出符合条件的元组。

优化:可以将选择操作提前到叉积之前,并且将叉积操作变为连接操作,这样可以减少连接的数量,也避免了大量的叉积操作,优化之后

	\begin{lstlisting}
	select shop_id,`name`,city from (select shop_id,`name`,city_id from shop where avg_price<20) as A inner join city using(city_id);  -- 约0.0015s
	\end{lstlisting}

可以看到查询时间缩短了很多,可以进一步优化,对avg_price添加索引,查询时间缩短到0.0013s左右,相比最初几乎快了一倍。

	\begin{lstlisting}
	select A.shop_id,A.`name` from shop as A,shop as B where A.`name`=B.`name` and A.city_id=1 and B.city_id=2;  -- 约0.03s
	\end{lstlisting}

这样的查询对于此次的数据库中没有实用性的意义,但在其他的数据库中会经常碰到,这个操作需要将两个表做积,shop又有1000条记录,因此时间耗费很大,用了0.03s。

优化:将查询操作提前到连接之前,并且用连接操作代替叉积操作,比如

	\begin{lstlisting}
	select A.shop_id,A.`name` from (select shop_id,`name` from shop where city_id=1) as A inner join (select shop_id,`name` from shop where city_id=2) as B using(`name`);  -- 约0.0015s
	\end{lstlisting}

查询效率提高非常多。由此可见,对于多个表的联合查询,除了对查询字段添加索引之外,还可以将查询操作提前到连接操作之前,并且避免使用直接的叉积操作,多用连接操作代替,可以大大提高查询效率。

\subsubsection{存在量词(exist)查询}
	
exists查询与in查询一起讨论,对于以下的查询语句

	\begin{lstlisting}
	1.select * from A where exists (select * from B where B.id = A.id);
	2.select * from A where A.id in (select id from B);
	\end{lstlisting}

	1会遍历A中每个元组,去查看exists条件是否满足,主要用到B中的索引。2则先查询子查询,再根据A中的索引去查看是否满足,在通常情况下,索引充足时,当A为大表时in效率较高,反之exists效率较高。

对于not exists和not in查询

	\begin{lstlisting}
	1.select * from A where not exists (select * from B where B.id = A.id);
	2.select * from A where A.id not in (select id from B);
	\end{lstlisting}

1查询同上,遍历A。而2此时在not in的查询则无法使用A中的索引,因此A的每条记录都要在B中遍历,此时not exists要比not in效率高

在实际操作中,应根据表中已有的索引情况以及各表的大小来选取exists与in
如:
	\begin{lstlisting}
	select city from city where not exists(select 1 from shop where avg_price<100 and shop.city_id=city.city_id);  -- 约0.0015s
	select city from city where city_id not in (select city_id from shop where avg_price<100);  -- 约0.002s

	select shop_id,`name` from shop where city_id in(select city_id from city where city.city='长沙');  -- 约0.001s
	select shop_id,`name` from shop where exists(select * from city where city.city='长沙' and city.city_id=shop.city_id);  -- 约0.01s
	\end{lstlisting}

\subsection{其他查询}

\subsubsection{向表中插入/删除记录}

	\begin{lstlisting}
	insert into shop(shop_id,`name`,city_id,avg_price) value(10000000,'计算机',1,50);
	delete from shop where `name`='计算机';
	\end{lstlisting}

当在city_id和avg_price上均不存在索引时插入时间最短。但如果在city_id或avg_price上存在索引的话插入时还需要根据插入的情况对索引进行调整,索引越多额外花费的时间就越多。删除操作首先需要找到所要删除的记录,因此删除时尽量通过搜索主键字段来删除,同样删除操作后也需要调整索引结构。所以在实际中应当权衡选择操作和插入/删除/修改操作,添加合适数量的索引,以达到最高的综合效率。

\subsubsection{聚集函数}
	
除上文讨论过的group by以外,常用的聚集函数还有max,min,count,avg,sum,对于min和max

	\begin{lstlisting}
	select max(avg_price),min(avg_price) from shop where small_cate='面包甜点';
	\end{lstlisting}

首先可以想到对搜索的字段加上索引,此时的查询速度有0.0006s左右,条件允许的情况下可建立(small_cate,avg_price)这样的索引,查询时间可以缩短到0.0002s。explain语句显示select tables optimized away,表示这个查询已经是最优情况。

	\begin{lstlisting}
	select count(*),sum(avg_price),avg(avg_price) from shop where small_cate='面包甜点';
	\end{lstlisting}

同上,可对(small_cate,avg_price)添加索引,不同于max和min,count,sum和avg必须浏览所有符合条件的元组,查询时间在0.0004s左右。explain语句显示使用了索引,并且计划读取149行数据。

\subsection{查询优化技巧总结}

\begin{enumerate}
	\item 避免 SELECT *,需要什么选取什么。
	\item 创建索引对于单值查询和范围查询有较好的效果。
	\item 对于多个表的联合查询,避免使用笛卡尔积,而选用连接操作。
	\item 可以利用explain查询语句的查询计划,来改进查询语句,获得更好的效果
	\item 对于多表查询,尽量早的完成选择和投影操作,之后再作连接
	\item 多用not exists代替not in,并根据实际情况选择exists或in
	\item 对于用到多个字段的查询,选择合适的索引顺序会使效率更高
\end{enumerate}

\section{Task 3: 数据库设计改进}

通过TASK2的查询优化分析,结合源数据的使用的场景,对数据库进行改进:

\begin{enumerate}
	\item 对name, small_cate_id, avg_price, stars, all_remarks等字段分别建立索引。
	\item 尽量使用可以正确存储数据的最小数据类型和简单类型。相比于实验一开始时随意选择的数据类型,由于更小、更简单的数据类型占用更小的磁盘、内存和CPU,所以改用现在TASK1的建表语句后,TASK2中查询的速度有了较大提升。
\end{enumerate}

\end{document}

